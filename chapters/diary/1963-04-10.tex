\fraktursection{Berlin, 10 April 1963}

Moons have passed in my hunt for Tatyana.
The Stasi's files yield no trace of her, and their cold efficiency, while a useful tool in my hands, is of no use when chasing a phantom.

Tonight, I was patrolling Karl-Marx-Allee.
I stood beneath the Frankfurter Tor, where the Party believes it sees everything.
But I have walked among their watchers and seen the truth: they do not need to see everything, nor do they.
They only need their people to believe they are seen.

I tried to read the thoughts of those around me, hoping for a glimpse of Tatyana, but their minds were closed, locked away behind walls of conformity.
All I could glimpse were their fears and desires, as predictable as the monotonous march of their propaganda.

This entire experience has been a lesson for my eventual return to Barovia.
The DDR has crushed the human spirit in this city more than I could ever imagine.
There are no mists here, yet the people have built borders and walls; a prison of their own making, which they willingly guard.
Everyone here is equal, yet they protect their untouchable lords.
I was a ruthless leader in my time, but I never witnessed such a profound level of self-imposed oppression.
This has convinced me that the world beneath the Wall is equally rotten.

The iron fist is not the only way to rule.
Give humans a reason — a cause, a dream, a fear — and they will build their own chains.
They will police themselves, denounce their neighbors, and offer their loyalty freely to the very hand that tightens the noose.
The Stasi does not need to hunt every dissident; it only needs to make the very idea of dissent unbearable.

In Barovia, I ruled through the mists, through the certainty that no one could leave.
Here, they rule through the mind, through the certainty that no one can think freely.
It is... elegant.
I will learn from it.

\vfill\newpage
