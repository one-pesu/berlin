\fraktursection{Berlin, 28 December 1962}

Today, I had to conduct my first interrogation.
The Stasi, it seems, is a tool of the state, a mechanism of control and surveillance.
It is clear that this is the most fitting place for me to begin my search for Tatyana.

The interrogation room was stark and cold, the walls bare except for a single light bulb hanging from the ceiling.
Manipulating him to reveal all the information Stasi needed was easy, but I had no interest in the mundane affairs of this world.
I needed to know about Tatyana, but without revealing too much about her to Stasi.
She is mine, and I will not share her with anyone.

I forced him to describe a woman that matched closely Tatyana's description, with a few subtle changes.
This was enough to grant me access to lists of names, addresses, and photographs, so I can look for her myself, while Stasi will be chasing shadows.

Upon my departure, the same enigmatic figure from before approached me again.
His ID badge was prominently displayed on his chest, reading ``Nikolai Volkov.''
I was not alone this time; I was accompanied by another Stasi agent, Genosse Friedrich.
I commanded him to leave, so I could speak with Klaus alone.
I felt Klaus' command over Friedrich, commanding him to stay.
I did not have the time for this duel.
I decided that there was much more to gain by allowing Klaus to reveal his intentions, so I stopped the command.

All three of us walked through the city, the agent oblivious to our nature.
Klaus took out a long pipe and lit it, then began to smoke.
He blew a cloud of smoke into the air and said, ``It looks like a mist, Genosse Strahd, does it not?''
As we walked past this cloud of smoke, smiling, he continued, ``But if it were a mist, it would not be so easy to escape, would it? Mists are treacherous, Genosse Strahd.
And perhaps, you never escape.
But you are exactly where they wanted you to be.''
It felt more like a warning than a statement, and I have spent the rest of the walk deep in my thoughts.
My gambit to allow him to keep Friedrich revealed no apparent advantage.
What was his game? Was he trying to manipulate me, or was he simply testing my resolve?

\subsection*{Reflections on Nikolai Volkov}

Nikolai Volkov is a variable I did not anticipate.
Is he trying to provoke me into a direct confrontation? I must tread lightly, for that would only bring the full, brutal weight of the DDR upon me.
He knows very well who I am; therefore, he should know that I will not be intimidated.
I will, however, not be reckless either.
He knows me, but I do not know him.
I must learn more about him, and quickly.
Whatever his intentions, I must be prepared.

And here is a gnawing thought: what if he is here for Tatyana? This is a possibility I cannot ignore.
If he is, then I must find her before he does.
Yet, there might be a temporary advantage to be gained by allowing him to believe we are in some way aligned.
That is a dangerous gambit, but one I must consider.
His knowledge, his power; they could be a tool.
I will not be used, but I can use him.
I will watch him, and I will find Tatyana first.

\vfill\newpage
