\fraktursection{Berlin, 28 December 1962}

Today, I had to conduct my first interrogation.
The Stasi, it seems, is merely a tool of the state, a colossal mechanism of calculated control and omnipresent surveillance.
This drab, efficient ministry is the most fitting place for me to begin my search for Tatyana.
It is a modern, secular imitation of my own castle, exchanging gothic horror for bureaucratic dread.

The interrogation room was stark and cold, an intentional vacuum.
The walls were bare, illuminated by a single, harsh light bulb hanging from the ceiling.
A table, two chairs, and a recording device completed the setup.
Everything about this place was meticulously designed to strip away any sense of individuality or self-worth before a word was even spoken.
Everything screamed intimidation and control—a method I am all too familiar with.

My subject was a low-level agitator, a man whose treason was born of petty desperation, not conviction.
Manipulating him to reveal all the information the Stasi needed was effortlessly easy.
The poor, frightened soul offered up his meager secrets like a child's toy.
Yet, I had no interest in the mundane affairs of this world, only in its records.
I needed to know about Tatyana, but without revealing the true target of my search to the Stasi apparatus.
She is mine, eternally and absolutely, and I will not share the object of my obsession with anyone.

I forced him, through subtle mental pressure, to describe a woman that matched Tatyana's true physical description, with a few crucial, subtle changes—a different eye color, a different hair shade.
This fabricated profile was enough to grant me official access to lists of names, addresses, and photographs.
The Ministry would be chasing a phantom, a decoy I manufactured, while I could sift through the raw data myself.

I left the sterile confines of the interrogation room and headed to the records room.
The fabricated need for follow-up intelligence gave me all the reason I needed to poke around, without raising suspicion.
A guard halted me before admitting entry.
He reeked of cheap, stale tobacco, cheap alcohol, and the lukewarm instant coffee that fuels this pathetic regime.
While he droned on with mandatory questions, I could not resist noticing the strong, pulsing vein of his jugular.
Yet, all those poisons he consumed made him utterly off-putting.
I reminded myself that I should feed before dawn, but not on such contaminated sustenance.

After a few boring minutes, I finally entered the archive.
The room was a monument to human fear, lined with endless shelves holding the compiled life of every human being in this sector: reduced to a number, a photo, and a clinical assessment.
This, I realized, was the true horror of their dominion.

Upon my departure, the same enigmatic figure from before approached me again.
His ID badge was prominently displayed on his chest, reading ``Nikolai Volkov.''
I was not alone this time; I was accompanied by another Stasi agent, Genosse Friedrich.

I commanded Friedrich to leave, intending to speak with Volkov alone, but I felt Volkov's silent command over the agent—a subtle psychic pressure that commanded him to stay.
I did not have the time for this duel of wills.
I decided that there was much more to gain by allowing him to reveal his intentions on his own terms, so I instantly ceased my command.

All three of us walked through the city, the oblivious agent serving as our unnecessary chaperone.
Volkov took out a long pipe, lit it with unnerving calm, and then began to smoke.
He blew a dense cloud of smoke into the night air.
``It looks like a mist, Genosse Strahd, does it not?'' he asked, a subtle, cold smile playing on his lips.

As we walked past this dissolving cloud, he continued, his voice dropping to a low, conspiratorial register.
``But if it were a mist, it would not be so easy to escape, would it?
Mists are treacherous, Genosse Strahd.
And perhaps, you never truly escape.
But you are exactly where they wanted you to be.''

It felt more like a pointed warning than a statement, and I have spent the rest of the walk deep in my thoughts, dissecting every word.
My initial gambit to allow him to keep Friedrich revealed no apparent advantage, save for his demonstration of power.
What was his true game? Was he trying to manipulate me into a rash move, or was he simply testing my patience?

\subsection*{Reflections on Nikolai Volkov}

Nikolai Volkov is a variable I did not anticipate, a disruptive element in a perfectly planned hunt.
Is he trying to provoke me into a direct, senseless confrontation? I must tread lightly, for a public display of my power would only bring the full, brutal weight of the DDR upon me—a confrontation I will not waste on him.
He knows very well who I am; therefore, he should know that I will not be intimidated.
I will, however, not be reckless either.

He knows me, but I do not know him.
I must learn more about him, and quickly.
Whatever his intentions, I must be prepared.

And here is the gnawing thought that festers in the stillness: what if he is here for Tatyana? This is a possibility I cannot ignore; it is the most likely, most tragic answer.
If he is, then I must find her before he does.
Yet, there might be a temporary advantage to be gained by allowing him to believe we are in some way aligned.
That is a dangerous gambit, requiring a performance of subservience I find repellent, but one I must consider.
His knowledge, his power—they could be a vital, if temporary, tool.
I will not be used, but I can certainly use him.
I will watch him, and I will find Tatyana first.
He may lead me to her, and then I will be rid of him.




\vfill\newpage
