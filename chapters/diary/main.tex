After a long day of fruitless interrogations and mounting paperwork, Dr. Weißmüller made himself a cup of strong, black coffee and returned to his office.
The hour was late, and the headquarters was quiet, save for the distant hum of ventilation and the occasional sharp ring of a telephone in another corridor.

His desk was immaculate, a stark contrast to the chaos of the Checkpoint Charlie case file.
Before him lay a standard-issue grey ministry folder.
Inside was the decrypted diary of Genosse Strahd von Zarovich.

A quick report from Moscow’s elite cryptographers at ГЛАВУАД was clipped to the front page, its contents as unsettling as the evidence from Strahd's apartment.
He read the summary with a critical eye.
The author, they concluded, had not used a known cipher.
He had developed a unique, archaic script and a personal language that seemed to be a dead dialect.
It was a linguistic nightmare.

Furthermore, they found a second, more complex layer of code embedded within the text.
That, Weißmüller surmised, would be the real intelligence: the lists of contacts, the mission parameters.

The attached pages, the report stated, were merely a ``direct, word-for-word translation of the primary narrative.''
The technicians had dismissed this surface-level text as ``fantastical'', ``psychologically inconsistent'', and ``likely a sophisticated diversionary construct.''

Dr. Weißmüller took a sip of the scalding coffee.
The cryptographers were technicians; he was an interrogator.
They saw a diversion; he saw a psychological profile.
This narrative, however absurd, was the key to understanding the mind of the operative known as Strahd.

Was this the elaborate, structured fantasy of a schizophrenic?
Or maybe a ``Munchausen-type story'' constructed by a deep-cover interrogator who had been pushed too far and needed to cope with the immense stress of his double life?
Or was it the meticulous, arrogant record of a traitor, cloaking his true activities in gothic nonsense?

He adjusted his desk lamp, its single cone of light illuminating the first typed page.
He settled in, ready to analyse the words of a madman, a traitor, or a monster.

\vfill\newpage
