\autoChapter

The aftermath of the mauling at Checkpoint Charlie was still raw, the screams of the dying now a bitter echo in the cold Berlin air.
By dawn's reluctant light, Sondervernehmer Dr. Karl-Heinz Weißmüller arrived at the scene.
The forensics team had completed their initial sweep, but the air hung heavy with the metallic tang of blood and cordite; a profane perfume that clawed at his nostrils.
The thick fog that had blanketed the area the night before had lifted, revealing a tableau of carnage that defied comprehension.

Dr. Weißmüller was a man of science; the rational mind was his compass.
He was determined to unravel the mystery with methodical precision.
He dismissed the witnesses' talk of monstrous shapes glowing in the fog, unearthly howls, and the sound of flapping wings as the fevered imaginings of traumatised civilians.
He scrutinised every spent bullet casing, every congealed drop of blood staining the barriers, every ragged scrap of torn fabric fluttering in the breeze.
The bullet-riddled walls bore witness to the frenzy, pockmarked craters whispering of desperate exchanges in the night.
Kneeling amid the fallen, he examined the bodies with clinical detachment, his gloved fingers tracing the savage bite marks gouged into their necks; deep, eerily precise punctures.
The corpses lay unnaturally serene, exsanguinated husks drained to pallid translucence, their faces contorted in rictus masks of abject terror.

A meticulous headcount confirmed the slaughter's totality: every soldier stationed at the checkpoint that fateful shift was accounted for among the dead, their uniforms sodden with the life they had spilled.
The two pursuing agents, von Zarovich and Volkov, who had plunged into the mist after the defectors, were nowhere to be found.

Dr. Weißmüller wasted no time.
Had they defected to the West?
Or, worse yet, were they traitors?
The Stasi could not afford to leave any stone unturned.
Time was of the essence.

He stormed Volkov's flat first, his boots echoing ominously in the narrow hallway.
He kicked the door open, pistol drawn and ready.
The room was almost empty, as if no one had ever truly lived there.
He inspected the bedroom, finding a bed neatly made, a small desk, and a single chair.
Not a single personal item was present, no photographs, no letters.
Even more curious, not a single speck of dust, as if the apartment were staged for an upcoming inspection.
He moved on to the living room, kitchen, and bathroom, each space a picture of startling, almost artificial perfection.
He found only a few berries from an unknown plant, which he swiftly bagged and tagged for analysis.
Opening a drawer, he discovered a knife with a sharp, pointed blade carved from oak.
The handle bore strange carvings, resembling an unknown alphabet.
He bagged and tagged it as well, then left his officers to search the flat thoroughly while he moved on to Strahd's residence.

He rushed through the crowd in Alexanderplatz, his mind replaying the night's events: a bloodbath at Checkpoint Charlie, a group of defectors now in the West, two highly trained agents, both missing.
The first residence was cleaner than a surgical suite.
His thoughts raced, searching for a logical explanation.

When he arrived at Strahd's residence, he found it even more curious.
It was already noon, but when he knocked down the door, it was pitch black inside.
Not a single ray of sunlight penetrated the gloom.
All the windows and the door leading to the balcony were veiled from within by weighty, lightless fabric.
The curtains were the thickest he had ever seen, meticulously crafted to maintain the darkness.
As soon as he opened them, the room was deluged in a searing brilliance.
It took time for his eyes to adjust, so he took a moment to draw a deep breath.
The air was cold and stale, carrying a faint, musty odour.
When his eyes finally adjusted, he saw the same clinical sterility.
The walls were bare, save for a mirror bearing a strange inscription in the same exotic alphabet as the knife.
He touched it, feeling the cold glass under his fingertips.
It felt unnaturally cold, as if it were absorbing the warmth from his skin.
Was the superstitious nonsense he had heard all day starting to get to him?
He shook his head, dismissing the thought.
In the centre of the room sat a coffin-shaped wooden box.
Curious, he opened the lid and noticed it was lined with Egyptian cotton sheets and a pillow.
A faint hollow suggested that someone had recently lain there.

Dr. Weißmüller noticed a desk and a chair in the corner of the room.
Looking closer, he saw a notebook on the desk, a diary of some sort.
He opened it and found it was filled with the same strange alphabet as the mirror.
While he could not read it, he could immediately tell it was a journal, with dates and entries.

Was Genosse Strahd a capitalist spy?
The product of an impeccable, high-level sleeper programme?
Was his flawless interrogation record the result of insider knowledge?
Dr. Weißmüller could only speculate.
The book was sent for expedited decryption by Moscow's elite, its secrets destined to be revealed.

Dr. Weißmüller left the flat, entering the midwinter chill of Berlin, with the notebook tucked securely under his arm.
He entered the Stasi headquarters and made his way to the interrogation rooms.
He heard the testimonies of the witnesses, their voices tinged with fear and confusion.
A monstrous creature, glowing eyes, wings, a swarm of bats.
The mass hysteria was palpable, but Dr. Weißmüller remained steadfast in his scepticism.
He could only hope that the diary, when decrypted, would provide the answers he sought.
